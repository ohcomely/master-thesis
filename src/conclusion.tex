\chapter{Conclusion and Future Work}
\label{ch:conclusion}

This thesis presented a comprehensive evaluation of sparse symmetric matrix reordering algorithms across diverse matrix structures and application domains. Through systematic benchmarking on the Fritz HPC cluster, we assessed many reordering algorithms, ranging from classical degree-based methods to newer graph partitioning and hypergraph-based techniques, across multiple performance metrics including fill-in reduction, computational efficiency, memory consumption, and parallelization potential.

Our evaluation reveals that advanced methods such as METIS, SCOTCH, and hypergraph based approaches consistently outperform classical algorithms like AMD and RCM in terms of fill-in reduction, achieving 30-60\% fewer fill-ins across most domains. Among the hypergraph methods, smaller partition sizes (HG-2 and HG-4) demonstrate superior effectiveness. Our custom METIS implementation with improved coarsening (METIS+IC) validates the benefits of enhanced coarsening strategies, producing results competitive with the best hypergraph methods. However, this improvement comes at a computational cost. Hypergraph methods require significantly more time (up to 270 seconds), while simpler methods like RCM complete in under one second, in our dataset.

In the course of this work, we implemented several efficient variants of existing algorithms. Our GPU-accelerated RCM implementation demonstrates substantial speedups over CPU-based versions for large matrices. The METIS+IC variant, incorporating refined coarsening strategies, achieves fill-in quality comparable to more complex hypergraph methods while maintaining reasonable execution times. We also explored hypergraph-based reordering methods that exploit block structures in matrices, showing promise particularly for circuit and graph problems. 

% \section*{Future Work}

Several promising directions remain for future investigation. Machine learning approaches, particularly graph neural networks (GNNs), show encouraging results on small matrices but require techniques like coarsening or hierarchical representations to scale to larger problems. Reordering algorithms for emerging architectures also warrant further research. While we implemented GPU-accelerated RCM, extending GPU implementations to nested dissection, minimum degree variants, and hypergraph methods might yield performance improvements, although seems challenging given the sequential nature of such problems. Finally, it is important to state that fill-in reduction is not the only optimization challenge for GPU accelerated direct solvers, as I/O has become a significant portion of the runtime, and future work could explore reordering strategies that optimize for I/O efficiency, among other factors beyond just fill-in minimization.


%Fill-in might not be final optimization challenge, I/O has becomes a good share of the runtime of the direct solvers. Maybe we need to optimize for something else
