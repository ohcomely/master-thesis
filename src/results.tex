\chapter{Results}
\label{ch:results}

In this chapter, we first describe our evaluation setup, which consists of the HPC cluster used for our experiments, the matrix datasets, and the algorithmic implementations evaluated. We then present a detailed analysis of various performance metrics, including execution time, memory usage, parallelization and matrix fill-in reduction. 


% In this chapter, you want to show that your implementation meets the objectives.
% Start by briefly describing the type of results you have collected and how these results relate to the objectives.

\section{Evaluation setup}

Everything that is evaluated is ran on Fritz, a high-performance computing cluster at NHR@FAU.

\subsection{Hardware Infrastructure}

Fritz is a parallel CPU cluster operated by NHR@FAU, featuring Intel Ice Lake and Sapphire Rapids processors with an InfiniBand (IB) network and a Lustre-based parallel filesystem accessible under \texttt{\$FASTTMP}. The cluster configuration consists of:
\begin{table}[h]
\centering
\begin{tabular}{|p{1.5cm}|p{4.5cm}|p{2cm}|p{3cm}|}
\hline
\textbf{Nodes} & \textbf{CPUs and Cores} & \textbf{Memory} & \textbf{Slurm Partition} \\
\hline
992 & 2 × Intel Xeon Platinum 8360Y (Ice Lake) & 256 GB & singlenode, multinode \\
  & 2 × 36 cores @2.4 GHz & & \\
\hline
48 & 2 × Intel Xeon Platinum 8470 (Sapphire Rapids) & 1 TB & spr1tb \\
   & 2 × 52 cores @2.0 GHz & & \\
\hline
16 & 2 × Intel Xeon Platinum 8470 (Sapphire Rapids) & 2 TB & spr2tb \\
   & 2 × 52 cores @2.0 GHz & & \\
\hline
\end{tabular}
\caption{Fritz cluster node configuration}
\label{tab:fritz-nodes}
\end{table}

The login nodes fritz[1-4] are equipped with 2 × Intel Xeon Platinum 8360Y (Ice Lake) processors and 512 GB main memory. Additionally, the remote visualization node fviz1 features 2 × Intel Xeon Platinum 8360Y processors with 1 TB main memory, one Nvidia A16 GPU, and 30 TB of local NVMe SSD storage.



% Precisely describe your measurement/evaluation setup in a top-down manner.
% For example:
% \begin{itemize}
%   \item If you used a development platform, which one and how was it configured?
%   \item Which tools did you use?
%   \item What input data set / program / signals / \ldots{} did you use?
%     If your data set is very large, you should fully define and describe it in an appendix chapter and refer to that definition here using simple labels.
%   \item If your implementation is configurable, which configuration(s) did you evaluate?
% \end{itemize}

The information given here (with references to external work) should be sufficient to reproduce the setup you used for your measurements.

\section{Other sections}

Structure the rest of this chapter into sections.
For example, each section could discuss a different figure of merit or a different part of your implementation.
As usual, discuss structure and content in advance with your advisors.

\section{Comparison to related work}

Compare your results to those achieved by others working on similar problems.
This can be done in a separate section or directly in the sections above.

\section{Current limitations}

Demonstrating that your implementation meets the objectives is usually done by showing a lower bound on a given figure of merit.
Conversely, you should also illuminate the limitations of your implementation by showing upper bounds on these (or other appropriate) figures of merit.
Critically examining your ideas and their implementation trying to find their limits is also part of your work!
